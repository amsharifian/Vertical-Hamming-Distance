To simplify searching a large database such as the human genome there are two major categories: seed-and-extend heuristic methods and suffix-array mapping methods. The \emph{seed-and-extend} heuristic is developed based on the observation that for a correct mapping, the short query read and its corresponding reference fragment, which is the piece of the reference genome that the query read should map to, must share some brief regions (usually 10-100 base-pair-long) of exact or inexact matches. These shorter shared regions, which indicate high similarity between the query read and the reference fragment, are called seeds. By identifying the seeds of a query read, the mapper narrows down the searching range from the whole genome to only the neighborhood region of each seed. Seeds are generated by preprocessing the reference genome and storing the locations of their occurrences in the reference genome in a separate data structure. During mapping, a seed-and-extend mapper first analyzes the query read to identify the seeds. Then, the mapper tries to extend the read at each of the seed locations via dynamic programming algorithms such as the Smith-Waterman \citep{smith1981identification} or Neddleman-Wunsch \citep{needleman} algorithm.
  
  