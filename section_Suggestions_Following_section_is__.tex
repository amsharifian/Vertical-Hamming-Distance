\section{Suggestions}

Following section is description of three suggestions about future works.

\subsection{Changing alphabet size}
Till now size of the alphabet was four ($\Sigma = \{A,C,G,T\}$) but can increase size of our alphabet by coding two characters instead of one character. By coding two characters our alphabet would change to:

\begin{equation}
\nonumber
\Sigma = \{ AA, AC, AG, AT, \dots, TC, TG, TT \}
\end{equation}

Following coding schema imposes multiple changes on our previous schema:

\begin{enumerate}
\item Data would compress. For instance if previous that our read's length was 100 we had to save 100 coded for each character but now, we can save 50 codes.
\item The cost of compressing data in this fashion is increasing length our coding. If we code our alphabet with 3 bits. Now we are coding our alphabet with 5 bits.
\item In vertical storage layout, if we transfer same size of bits in each iteration, for instance 16 bits, we are transferring bigger portion of the read (instead of $16\%$ of the code we are transferring $32\%$ of the code.
\item Our filter's sensitivity will \emph{decrease}. In the first schema we could detect every substitution of, for instance, T to A, C and G. But now we are detecting all substitiution of AT to rest of permutation of characters.
\end{enumerate}
  
  
  
  